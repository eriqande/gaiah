\documentclass[11pt,landscape]{article}
\usepackage{graphicx}
\usepackage{amssymb}
\usepackage{epstopdf}
\usepackage{amsfonts}
\usepackage{natbib}
\usepackage{subfigure}
\usepackage{pdfsync}
\usepackage{xspace}
%\usepackage[pdftex, colorlinks=true,urlcolor=blue]{hyperref}
%\usepackage{wallpaper}

%%%% HERE IS SOME WATERMARK STUFF
%\addtolength{\wpXoffset}{-.2in}
%\CenterWallPaper{1.1}{/Users/eriq/Documents/work/nonprj/WaterMarks/StrictDraftWatermark.eps}


\DeclareGraphicsRule{.tif}{png}{.png}{`convert #1 `dirname #1`/`basename #1 .tif`.png}



%% some handy things for making bold math
\def\bm#1{\mathpalette\bmstyle{#1}}
\def\bmstyle#1#2{\mbox{\boldmath$#1#2$}}
\newcommand{\thh}{^\mathrm{th}}


%% Some pretty etc.'s, etc...
\newcommand{\cf}{{\em cf.}\xspace }
\newcommand{\eg}{{\em e.g.},\xspace }
\newcommand{\ie}{{\em i.e.},\xspace }
\newcommand{\etal}{{\em et al.}\ }
\newcommand{\etc}{{\em etc.}\@\xspace}



%% the page dimensions from TeXShop's default---very nice
\textwidth = 9 in
\textheight = 7.0 in
\oddsidemargin = 0.0 in
\evensidemargin = 0.0 in
\topmargin = 0.0 in
\headheight = 0.0 in
\headsep = 0.0 in
\footskip = -0.2 in
\parskip = 0.2in
\parindent = 0.0in


\title{Supporting Information 1 for paper\\
``Tracking Migratory Animal Origins Using \\
Genetics, Isotopes, and Habitat Suitability''}
\author{Kristen C. Ruegg \and  
         Eric C. Anderson \and 
         Kristina L. Paxton\and 
         Ryan J. Harrigan \and
         Jeff Kelly \and
         Thomas B. Smith
         }

\begin{document}

\maketitle


This document contains supporting information for the paper, ``Tracking Migratory Animal Origins Using
Genetics, Isotopes, and Habitat Suitability.''  Each page shows four rows of plots.  Each row corresponds
to a single bird from the reference data set.  The bird's name and region of origin are given above each plot,
along with a nearby town, the latitude and longitude, and the bird's $\delta^2\mathrm{H}$ reading.  

Each row contains four plots. The first shows the prior probability of bird origin obtained from habitat suitability.
The second shows the posterior probability of origin given only the genetic data (and a uniform prior).  
The third shows the posterior probability of origin given only the isotope data (and a uniform prior).
The fourth shows
the posterior probability when both the genetics and the isotopes are used, and the habitat suitability is used as a prior.  

The prior from habitat suitability is identical for every bird, but may appear different due to differences in scaling of the 
colors used. A key appears at the right of every figure.
\newpage

\input{body_supp1.tex}

\bibliography{anderson}
\bibliographystyle{mychicago}
 \end{document}
